\documentclass[a4paper,latin,center,twocolumn]{paper} 
\usepackage[subpreambles=true]{standalone} % for separate files
\usepackage[english]{babel}  
\usepackage[margin=2.5cm]{geometry}
\usepackage{graphicx}
\usepackage{lipsum}
\usepackage{xcolor}
\usepackage{booktabs}

\sectionfont{\large\sf\bfseries\color{black!70!blue}} 
% \renewcommand\keywordname{Trial Keyword}
\title{ME748: Computer Aided Simulation of Machines}
\subtitle{Exemplum apparentia \texttt{paper} tabellae}
\author{Ph. D. Franciscus Studiosum Somniantis} 
\institution{Ignotum Universitas \\ 
Ad requiem centrum Scientiarum}


\begin{document} 
\selectlanguage{english}
\twocolumn[\maketitle 
    \hrule 
    \bigskip
]

\begin{abstract} 
    {\lipsum[12]} 
\end{abstract}

\begin{keywords}
    \raggedright MWE, \LaTeX, document class, \texttt{paper}, \texttt{article}, dummy text    
\end{keywords}

\smalltableofcontents

\section{Introduction} 
    Some introduction. \lipsum[2]
    Some intro from separate file.
\subsection{Subintro}
    \lipsum[2]
    % input just puts the text in the other file here.
    % \include{filename} essentially does a \clearpage before and after. \include is commonly used to put each chapter of a book or thesis into its own file.
    % \subsection[Self Intro]{SubIntro}
    %     \lipsum[2]
    
\section{Materia et modos} 
    \begin{figure}[hbt!]
        \centering
        \includegraphics[width=0.9\columnwidth]{Images/Indian_Institute_of_Technology_Bombay_Logo.png}
        \caption{Logo of IIT Bombay}
        \label{fig:IITBLogo}
    \end{figure}

    Please see Figure ~\ref{fig:IITBLogo} on page ~\pageref{fig:IITBLogo} for a prototype blah blah blah. More dummy text. \lipsum[4] 

\section{Consequitur} 
    These are the results. \lipsum[1]

    \begin{table}[hbt!] % hbt! is to force the thing on the same page; other options (things inside []) are b for bottom and nothing for default
        \centering
        \begin{tabular}{lcccccc}
            \toprule
            & I &  II & III & IV & V & VI \\
            \midrule
            Vandali     & 123 & 456 & 678 & 321 & 644 & 768  \\ 
            Visigoth & 021 & 229 & 678 & 123 & 456 & 678 \\     
            \bottomrule
        \end{tabular}
        \caption{\raggedright Visigothi cum Romanis foederati Hispaniam ingressi sunt et contra Vandalos. Mortem comitis utraque pugna.} 
    \end{table}
    
\section{Disputatio} 
    \lipsum[4] 

\section{Conclusionibus}
    \lipsum[5]

\end {document}