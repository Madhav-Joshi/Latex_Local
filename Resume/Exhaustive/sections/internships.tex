\section{Internships and Research}

\cventry{May'22 \\ Jul'22}{Modelling and Control of Compliant Mechanisms {\textbar} Research Internship}{}{}{}{\textit{Guide:} \href{https://www.linkedin.com/in/abhijeet-joshi-a2411718/}{\textit{Dr. Abhijeet Joshi}} $\vert$ \textit{\href{https://new.siemens.com/in/en/company/about.html}{Siemens Technology India}}\\
\vspace{-10pt}
\begin{itemize}
    \item Examined open source \textbf{SOFA framework} targeting real time for interfacing mesh, solvers, collision models and material properties to simulate a soft robot
    \item Implemented a research paper on \textbf{system modelling} using Koopman operator and \textbf{MPC control} of \textbf{soft robot} on compliant springs using strings coupled with brushed DC motors
    \item Hardware utilized \textbf{Raspberry Pi 4B} as controller and to interface camera, Arduinos and a PWM driver (PCA9685) using \textbf{I}\boldmath{\(2\)}\textbf{C} communication protocol
    \item Installed \textbf{ROS} on Raspbian OS and developed a ROS package which could be expanded to more complex robots whose controller achieved an accuracy of \boldmath{\(\pm 3\)} \textbf{mm}
\end{itemize}}

\cventry{Jan'21 \\ Jul'21}{Topology Optimization {\textbar} iSURP - In Semester UG Research Program}{}{}{}{\textit{Guide: \href{https://www.aero.iitb.ac.in/home/people/faculty/amuthan}{Prof Amuthan A Ramabathiran} {\textbar} Aerospace Department, IIT Bombay}\\
\vspace{-10pt}
\begin{itemize}
    \item Analyzed \textbf{density based} Topology Optimization in context of linear elasticity and applied it to a heat sink design optimization problem using open source \textbf{FEniCS} project in Python
    \item Implemented various numerical methods: gradient descent, forward/backward Euler, FDM and FEM in Python
    \item Formulated primal and adjoint equations for \textbf{Poisson membrane} problem to calculate the derivative of objective function using advanced analytical methods in variational calculus for \textbf{constrained optimization} problem
\end{itemize}}

\cventry{Dec'20 \\ Jan'21}{Microwave Metal Heating - 3D Modelling}{}{}{}{\textit{Guide:} \href{https://technology.nirmauni.ac.in/author/shrutimehta/}{\textit{Prof Shruti Bhatt}} {\textbar} \textit{Mechanical Engineering Dept} {\textbar} \textit{\href{https://nirmauni.ac.in/}{Nirma University}, Gujrat, India}\\
\vspace{-10pt}
\begin{itemize}
    \item Modelled \textbf{transient electromagnetic heating} of AA6061 specimen using COMSOL 3.5a
    \item Calculated time required for reaching melting temperature and verify using experimental data with \boldmath{\(95\%+\)} \textbf{accuracy} to conclude that microwave metal casting process is more efficient than conventional casting methods
\end{itemize}}
