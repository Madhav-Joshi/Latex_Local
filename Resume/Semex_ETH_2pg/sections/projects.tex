\section{Projects}

\cventry{Jan'23 \\ Feb'23}{JLR Robot Charging Challenge {\textbar} Inter IIT Tech Meet 11.0}{}{}{}{\textit{Bagged the \textbf{$1^{st}$ runner up} out of a total of 10+ teams from various IIT's across India}\\
\vspace{-10pt}
\begin{itemize}
    \item  The problem statement prescibed us to devise a system that can automatically detect the charging port of the vehicle and plug the socket into the charging port
    \item Designed a 6 DoF robotic arm using Solidworks for maximum range of motion without self collision and simulated it on MATLAB and Simscape in a team of 13
    \item Computed system trajectory to move the end effector using analytical inverse kinematics and respecting constraints like self-collisions, joint limits, and velocity limitations
    \item Performed torque and energy calculations to respect the torque limits on joint motors and compare energy consumed for traversing different trajectories of the robotic arm
\end{itemize}}

\cventry{Feb'22 \\ May'22}{H-Bot 3D Printer}{}{}{}{\textit{Course: Design of Mechatronic Systems} {\textbar} \textit{Guide: \href{https://www.me.iitb.ac.in/~gandhi/}{Prof. Prasanna Gandhi}, IIT Bombay}\\
\vspace{-10pt}
\begin{itemize}
    \item Constructed an H-Bot 3D printer mechanism using TIVA C microcontroller, stepper motors, timing pulleys, aluminum extrusions, and LSM13 linear encoder
    \item Implemented a subroutine using PWM signals to drive two stepper motors simultaneously, QEI to read linear encoder values, and take the station to the desired input position 
\end{itemize}}

\cventry{Feb'22 \\ May'22}{Temperature Control of Fluid Column}{}{}{}{\textit{Course: Advanced Feedback Theory} {\textbar} \textit{Guide: \href{https://scholar.google.co.in/citations?user=Sm0PDg8AAAAJ&hl=en}{Prof. P.S.V. Nataraj}, \href{https://www.sc.iitb.ac.in/}{SysCon}, IIT Bombay}\\
\vspace{-10pt}
\begin{itemize}
    \item Conducted bump test on an integrating system and collected the data from National Instruments DAQ card using MATLAB interface for modelling purposes
    \item Tuned and tested a PI controller based on Ziegler Nichols tuning rules, and integrated a feed forward block to enhance  the disturbance rejection for the system
\end{itemize}}

\cventry{Feb'22 \\ May'22}{Vision Transformer on Small Data-set}{}{}{}{\textit{Course: Machine Learning for Remote Sensing} {\textbar} \textit{Guide: \href{https://scholar.google.it/citations?user=IEcsMPAAAAAJ&hl=en}{Prof. Biplab Banerjee}, IIT Bombay}\\
\vspace{-10pt}
\begin{itemize}
    \item Demonstrated proficiency in implementing state-of-the-art (SOTA) ViT and SL-ViT models for image classification on CIFAR-10 and CIFAR-100 to increase locality inductive bias
    \item Enhanced locality inductive bias through the use of Shifted Patch Tokenization and Locality Self Attention techniques, resulting in more spatial information and locally focused attention
    \item Achieved top-5 accuracy of 99.13\% for CIFAR-10 and 82.92\% for CIFAR-100 dataset
\end{itemize}}

\cventry{Feb'22 \\ May'22}{Two Degrees of Freedom Robotic Arm}{}{}{}{\textit{Course: Robotics} {\textbar} \textit{Guide: \href{https://scholar.google.co.in/citations?user=EI2hj8gAAAAJ&hl=en}{Prof. Abhishek Gupta} {\textbar} Mechanical Engineering, IIT Bombay}\\
\vspace{-10pt}
\begin{itemize}
    \item Designed and constructed a 2 DoF Robotic Arm with a plane movement capability, based on a CAD model created in Solidworks that comprised of PLA links and servo motors at joints
    \item Implemented hardware and software integration by developing a red spot detection algorithm and integrating the system with an Arduino for closed loop feedback control
    \item 3D printed all the components required for assembling while ensuring accuracy and precision
    \item Successfully achieved accurate joint parameters, and current and desired positions of the end-effector, resulting in the precise movement of the robot arm to coordinates pointed by a red spot detected by a phone camera around the robot
\end{itemize}}

\cventry{Jul'21 \\ May'22}{Stride {\textbar} Student Technical Team}{}{}{}{\textit{Focuses on building a quadruped which can walk autonomously on all terrain} {\textbar} \textit{IIT Bombay}\\
\vspace{-10pt}
\begin{itemize}
    \item Developed the Newton-Euler method in MATLAB for calculating the reaction forces and torques needed to achieve desired angular velocities and accelerations in any configuration
    \item Reviewed the MIT Cheetah 3 papers,  robot design, including examining its implementation of MPC with QP formulation in ROS and simulating the robot in Gazebo and RViz
    \item Conducted extensive research on trajectory generation techniques using Bezier curves and the localization and mapping of quadruped robots in diverse environments.
\end{itemize}}

\cventry{Mar'22 \\ Apr'22}{Optimization of Swiggy Instamart Hub Locations}{}{}{}{\textit{Course: IEOR} {\textbar} \textit{Guide: \href{https://www.me.iitb.ac.in/?q=faculty/Prof.\%20Avinash\%20Bhardwaj}{Prof. Avinash Bharadwaj} {\textbar} Mechanical Engineering, IIT Bombay}\\
\vspace{-10pt}
\begin{itemize}
    \item Formulated Swiggy Instamart's revenue system as an \textbf{integer programming problem} by considering investment costs, operating costs and profits from all probable hub locations
    \item Optimized net profit for both limited and infinite capacity hub scenarios using CPLEX solver in \textbf{AMPL}. Ensured model robustness by conducting uncertainty and risk analysis
\end{itemize}}

\cventry{May'21}{Laser Surface Hardening (LSH)}{}{}{}{\textit{Course: Manufacturing Processes} {\textbar} \textit{Guide: \href{https://www.me.iitb.ac.in/~ramesh/}{Prof Ramesh K Singh}, Mechanical Department, IITB}\\
\vspace{-10pt}
\begin{itemize}
    \item Simulated LSH process via \textbf{FEM} using open-source \textbf{FEniCS} project with a \textbf{team of 5}
    \item Reviewed 5+ research papers and numerically solved \textbf{transient heat equation} to calculate temperature field and found the \textbf{laser velocity} for required \textbf{hardened depth} and vice versa
\end{itemize}}