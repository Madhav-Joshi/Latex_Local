%%%%%%%%%%%%%%%%%%%%%%%%%%%%%%%%%%%%%%%%%
% "ModernCV" CV and Cover Letter
% LaTeX Template
% Version 1.3 (29/10/16)
%
% This template has been downloaded from:
% http://www.LaTeXTemplates.com
%
% Original author:
% Xavier Danaux (xdanaux@gmail.com) with modifications by:
% Vel (vel@latextemplates.com)
%
% License:
% CC BY-NC-SA 3.0 (http://creativecommons.org/licenses/by-nc-sa/3.0/)
%
% Important note:
% This template requires the moderncv.cls and .sty files to be in the same 
% directory as this .tex file. These files provide the resume style and themes 
% used for structuring the document.
%
%%%%%%%%%%%%%%%%%%%%%%%%%%%%%%%%%%%%%%%%%

%----------------------------------------------------------------------------------------
%	PACKAGES AND OTHER DOCUMENT CONFIGURATIONS
%----------------------------------------------------------------------------------------

\documentclass[11pt,a4paper,roman,unicode]{moderncv} % Font sizes: 10, 11, or 12; paper sizes: a4paper, letterpaper, a5paper, legalpaper, executivepaper or landscape; font families: sans or roman

\moderncvstyle{classic} % CV theme - options include: 'casual' (default), 'classic', 'oldstyle' and 'banking'
\moderncvcolor{blue} % CV color - options include: 'blue' (default), 'orange', 'green', 'red', 'purple', 'grey' and 'black'

% \usepackage{lipsum} % Used for inserting dummy 'Lorem ipsum' text into the template
\usepackage[left=2cm, top=2cm, right=2cm, bottom=2cm]{geometry} % Reduce document margins
%\setlength{\hintscolumnwidth}{3cm} % Uncomment to change the width of the dates column
%\setlength{\makecvtitlenamewidth}{10cm} % For the 'classic' style, uncomment to adjust the width of the space allocated to your name

%----------------------------------------------------------------------------------------
%	NAME AND CONTACT INFORMATION SECTION
%----------------------------------------------------------------------------------------

%\firstname{Madhav} % Your first name
%\familyname{Joshi} % Your last name
\name{Madhav}{Joshi}

\renewcommand*{\namefont}{\fontsize{58}{30}\mdseries\upshape}

% All information in this block is optional, comment out any lines you don't need
%\title{Curriculum Vitae}
\address{Department of Mechanical Engineering,}{Indian Institute of Technology Bombay,\\Powai, Mumbai, Maharashtra (400076), India\\}
%\mobile{(+91) 6364030375}
%\phone{(000) 111 1112}
%\fax{(000) 111 1113}
%\email{madhav.goj@gmail.com $|$ 190110034@iitb.ac.in}
%\extrainfo{\emailsymbol\emaillink{190110034@iitb.ac.in}}
\renewcommand{\emaillink}[1]{#1} % prevent a non-sensical mailto link
\email{\href{mailto:madhav.goj@gmail.com}{madhav.goj@gmail.com}\makenewline\emailsymbol{}\href{mailto:190110034@iitb.ac.in}{190110034@iitb.ac.in}}
\social[linkedin][https://www.linkedin.com/in/madhav-j/]{madhav-j}
%\homepage{https://github.com/Madhav-Joshi}{Github: Madhav-Joshi} % The first argument is the url for the clickable link, the second argument is the url displayed in the template - this allows special characters to be displayed such as the tilde in this example
%\extrainfo{additional information}
%\photo[70pt][0.4pt]{pictures/picture} % The first bracket is the picture height, the second is the thickness of the frame around the picture (0pt for no frame)
%\quote{"A witty and playful quotation" - John Smith}

%----------------------------------------------------------------------------------------

\begin{document}

%----------------------------------------------------------------------------------------
%	COVER LETTER
%----------------------------------------------------------------------------------------

% To remove the cover letter, comment out this entire block

%\clearpage

%\recipient{HR Department}{Corporation\\123 Pleasant Lane\\12345 City, State} % Letter recipient
%\date{\today} % Letter date
%\opening{Dear Sir or Madam,} % Opening greeting
%\closing{Sincerely yours,} % Closing phrase
%\enclosure[Attached]{curriculum vit\ae{}} % List of enclosed documents

%\makelettertitle % Print letter title

%\lipsum[1-2] % Dummy text
%\lipsum[4] % Dummy text

%\makeletterclosing % Print letter signature

%\newpage

%----------------------------------------------------------------------------------------
%	CURRICULUM VITAE
%----------------------------------------------------------------------------------------

\makecvtitle % Print the CV title

%----------------------------------------------------------------------------------------
%	EDUCATION SECTION
%----------------------------------------------------------------------------------------
\vspace{-10mm}
\section{Education}

\cventry{2019--2024}{\href{http://www.iitb.ac.in}{Indian Institute of Technology Bombay}, Mumbai, India}{}{}{}{\begin{itemize}
    \item Major: \href{https://www.me.iitb.ac.in/}{\textbf{Mechanical Engineering}}\\
    GPA: \textbf{9.27/10 (Department Rank: 2 out of 40)}
    \item Minor: \href{https://www.sc.iitb.ac.in/}{\textbf{Systems and Controls Engineering}}\\
    GPA: \textbf{9.25/10}
\end{itemize}}  % Arguments not required can be left empty
\vspace{-3mm}

\section{Awards and Achievements}

\cvitem{2022}{Awarded \textbf{Institute Academic Award} for performing \(\mathbf{2^{nd}}\) best for a whole year}
%
\cvitem{2019}{Among \textbf{Top 1.5\%} out of 200,000 students in JEE Advanced 2019}
%
\cvitem{2019}{Secured \textbf{99.54 percentile} among 1.15M candidates in \textbf{JEE Mains 2019}}
%
\cvitem{2019}{Secured \textbf{All India Rank} \boldmath{\(74\)} among 15,000 students in \textbf{UCEED} (Design Entrance)}
%
\cvitem{2019}{\textbf{Ranked} \boldmath{\(107^{th}\)} out of 0.39 million examinees in Maharashtra Common Entrance Test}

%----------------------------------------------------------------------------------------
%	WORK EXPERIENCE SECTION
%----------------------------------------------------------------------------------------
\vspace{-3mm}

\section{Internships and Technical Teams}

\cventry{May'22-Jul'22}{Modelling and Control of Compliant Mechanisms $|$ Research Internship}{}{}{}{\textit{Guide:} \href{https://www.linkedin.com/in/abhijeet-joshi-a2411718/}{\textit{Dr. Abhijeet Joshi}} $\vert$ \textit{\href{https://new.siemens.com/in/en/company/about.html}{Siemens Technology India}}\\
\vspace{-10pt}
\begin{itemize}
    \item Examined open source \textbf{SOFA framework} targeting real time for interfacing mesh, solvers, collision models and material properties to simulate a soft robot
    \item Implemented a research paper on \textbf{system modellig} using Koopman operator and \textbf{MPC control} of \textbf{soft robot} on compliant springs using strings coupled with brushed DC motors
    \item Hardware utilized \textbf{Raspberry Pi 4B} as controller and to interface camera, Arduinos and a PWM driver (PCA9685) using \textbf{I}\boldmath{\(2\)}\textbf{C} communication protocol
    \item Installed \textbf{ROS} on Raspbian OS and developed a ROS package which could be expanded to more complex robots whose controller achieved an accuracy of \boldmath{\(\pm 3\)} \textbf{mm}
\end{itemize}}

\cventry{Jul'21 -- May'22}{Stride $|$ Student Technical Team}{}{}{}{\textit{Focuses on building a quadruped which can walk autonomously on all terrain} $\vert$ \textit{IIT Bombay}\\
\vspace{-10pt}
\begin{itemize}
    \item Coded Newton-Euler method in \textbf{MATLAB} to find the reaction forces and torques required to produce desired angular velocities and acceleration in any configuration
    \item Reviewed \textbf{MIT Cheetah 3 robot} design, \textbf{MPC} implementation using QP formulation in \textbf{ROS} and tried to simulate Gazebo and RViz
    \item Researched trajectory generation using Bezier curves, localization and mapping of the quadruped in the given environment
\end{itemize}}

\cventry{Dec'20 -- Jan'21}{Microwave Metal Heating - 3D Modelling}{}{}{}{\textit{Guide:} \href{https://technology.nirmauni.ac.in/author/shrutimehta/}{\textit{Prof Shruti Bhatt}} $\vert$ \textit{Mechanical Engineering Dept} $\vert$ \textit{\href{https://nirmauni.ac.in/}{Nirma University}, Gujrat, India}\\
\vspace{-10pt}
\begin{itemize}
    \item Modelled \textbf{transient electromagnetic heating} of AA6061 specimen using COMSOL 3.5a
    \item Calculated time required for reaching melting temperature and verify using experimental data with \boldmath{\(95\%+\)} \textbf{accuracy} to conclude that microwave metal casting process is more efficient than conventional casting methods
\end{itemize}}

%-----------------------------------------------

%----------------------------------------------------------------------------------------
%	Research Publication SECTION
%----------------------------------------------------------------------------------------
\vspace{-3mm}
\section{Research Publications}
\cvitem{}{Shruti, C. B., Nilesh, D. G., \& \textbf{Madhav, J.} (2021). Multiphysics simulation and validation of microwave melting characteristics of AA6061 by finite element analysis. \textit{Advances in Materials and Processing Technologies}. \href{https://doi.org/10.1080/2374068X.2021.1948708}{doi:10.1080/2374068X.2021.1948708}}

%-----------------------------------------------

%----------------------------------------------------------------------------------------
%	Projects SECTION
%----------------------------------------------------------------------------------------
%\vspace{10mm}
\newpage 

\section{Projects}
\cventry{Feb'22 -- May'22}{Two Degrees of Freedom Robotic Arm}{}{}{}{\textit{Course: Robotics} $|$ \textit{Guide: \href{https://scholar.google.co.in/citations?user=EI2hj8gAAAAJ&hl=en}{Prof. Abhishek Gupta} $|$ Mechanical Engineering, IIT Bombay}\\
\vspace{-10pt}
\begin{itemize}
    \item  Built a 2 Degree-of-Freedom Robotic Arm which moves in a plane to coordinates pointed by a red spot detected by phone camera around the bot
    \item Designed a \textbf{CAD model} for the robot in \textbf{Solidworks} consisting of Servo motors, joints and links of robot to \textbf{3-D print} all components required to assemble the robot
    \item Developed red spot detection algorithm, Forward and Inverse Kinematics in \textbf{MATLAB} for calculating joint parameters, current and desired position of end-effector
    \item Integrated \textbf{Arduino} with MATLAB for implementing \textbf{closed loop feedback control}
\end{itemize}}

\cventry{Mar'22 -- Apr'22}{Optimization of Swiggy Instamart Hub Locations}{}{}{}{\textit{Course: IEOR} $|$ \textit{Guide: \href{https://www.me.iitb.ac.in/?q=faculty/Prof.\%20Avinash\%20Bhardwaj}{Prof. Avinash Bharadwaj} $|$ Mechanical Engineering, IIT Bombay}\\
\vspace{-10pt}
\begin{itemize}
    \item Modelled the revenue system of Swiggy Instamart as an \textbf{integer programming problem} by considering investment costs, operating costs and profits from all probable hub locations
    \item Optimized the net profit for the cases where hubs have limited or infinite capacity to store the products using CPLEX solver in \textbf{AMPL}
    \item Checked model robustness by performing uncertainity and risk analysis
\end{itemize}}

\cventry{Aug'21 -- Nov'21}{Foot Pump Manufacturing Design}{}{}{}{\textit{Course: Manufacturing Processes} $|$ \textit{Guide: \href{https://www.me.iitb.ac.in/?q=faculty/Prof.\%20Deepak\%20Marla}{Prof. Deepak Marla} $|$ Mechanical Engineering, IITB}\\
\vspace{-10pt}
\begin{itemize}
    \item Studied manufacturing aspects of foot-pump including \textbf{Material and Process selection, Design and Inspection}
    \item Modelled every part of the pump and assembled them in \textbf{Solidworks}
    \item Carried out \textbf{cost analysis} based on the pump dimensions, materials and process used
\end{itemize}}

\cventry{Jan'21 -- Jul'21}{Topology Optimization $|$ iSURP - In Semester UG Research Program}{}{}{}{\textit{Guide: \href{https://www.aero.iitb.ac.in/home/people/faculty/amuthan}{Prof Amuthan A Ramabathiran} $|$ Aerospace Department, IIT Bombay}\\
\vspace{-10pt}
\begin{itemize}
    \item Analyzed \textbf{density based} Topology Optimization in context of linear elasticity and applied it to a heat sink design optimization problem using open source \textbf{FEniCS} project in Python
    \item Implemented various numerical methods: gradient descent, forward/backward Euler, FDM and FEM in Python
    \item Formulated primal and adjoint equations for \textbf{Poisson membrane} problem to calculate the derivative of objective function using advanced analytical methods in variational calculus for \textbf{constrained optimization} problem
\end{itemize}}

\cventry{Jun'21 -- Jul'21}{Algorithmic Trading $|$ FinSearch}{}{}{}{\textit{Finance Club of IIT Bombay}\\
\vspace{-10pt}
\begin{itemize}
    \item Grasped stock market knowledge from \textbf{financial modules} of Zerodha Varsity on basics of stock market, technical and fundamental analysis, Futures \& Options and their strategies, trading systems like pairs trading
    \item Scouted various trading strategies and implemented \textbf{pairs trading} strategy after back-testing results in Python and were among \textbf{top 6 teams} awarded with a cash prize of Rs 5 K for exemplary performance
\end{itemize}}

\iffalse
\cventry{May'21}{Laser Surface Hardening (LSH)}{}{}{}{\textit{Course: Manufacturing Processes} | \textit{Guide: \href{https://www.me.iitb.ac.in/~ramesh/}{Prof Ramesh K Singh} | Mechanical Department, IIT Bombay}\\
\vspace{-10pt}
\begin{itemize}
    \item Analysed and simulated LSH process via \textbf{FEM} using open-source \textbf{FEniCS} project collaborating in a \textbf{team of 5}
    \item Reviewed 5+ research papers and numerically worked out \textbf{transient heat equation} to calculate temperature field and developed a model to find the \textbf{laser velocity} for required \textbf{hardened depth} and vice versa
\end{itemize}}
\fi

\cventry{Jun'20}{Analemma $|$ Krittika Summer Project}{}{}{}{\textit{Krittika - Astronomy Club of IIT Bombay}\\
\vspace{-10pt}
\begin{itemize}
    \item Inspected the motion of sun using different \textbf{celestial coordinate systems} and developed a python code to generate the data points of Analemma and calculated its various properties with \textbf{98\% accuracy} 
    \item Generated \textbf{2D} and \textbf{3D interactive plots} and \textbf{colormaps} of Analemma and its properties using Matplotlib and calculated its properties like shape, size and position to study their variation with the orbital parameters both qualitatively and quantitatively
\end{itemize}}

\cventry{Apr'20}{Frequency Analysis of Linear Systems}{}{}{}{\textit{Controls and Dynamical Systems Group of IIT Bombay}\\
\vspace{-10pt}
\begin{itemize}
    \item Implemented MATLAB codes to design \textbf{low pass} and \textbf{high pass} filters in frequency domain to study the trade-off between smoothness and sharpness while filtering noise from an image
    \item Analysed the use of \textbf{frequency domain} in \textbf{spatial} and \textbf{time} domain analysis of systems by applying the knowledge of Signals and Feedback Systems including Time-Frequency analysis, Filters and Convolution
\end{itemize}}

\iffalse
\cventry{Sep'19}{Remote Controlled Plane}{}{}{}{\textit{Aeromodelling club of IIT Bombay}\\
\vspace{-10pt}
\begin{itemize}
    \item Fabricated fuselage, stabilizers and wings of a miniature aircraft to \textbf{reduce drag} and improve flight time
    \item Optimized the \textbf{performance} and \textbf{stability} of flight by careful consideration of design parameters
\end{itemize}}

\cventry{Aug'19}{XLR8 | Obstacle Manoeuvring Bot}{}{}{}{\textit{Electronics and Robotics Club of IIT Bombay}\\
\vspace{-10pt}
\begin{itemize}
    \item Designed and built a \textbf{Mobile App Controlled Bot} having Differential Steering using Bluetooth module to overcome various hurdles in the least possible time among 450+ participants
    \item Implemented the electrical sub-system for RC car utilizing \textbf{AT Tiny} microcontroller for controlling the bot
\end{itemize}}
\fi

%-----------------------------------------------

%----------------------------------------------------------------------------------------
%	Key Courses and Technical Skills SECTION
%----------------------------------------------------------------------------------------
\vspace{-3mm}


\section{Key Courses and Technical Skills}
\cvitem{Automation}{Mechatronics, Robotics, Vibro-Acoustics, Machine Learning, Advanced PID Control, Linear and Non Linear systems, Signals and Feedback systems}
\cvitem{Labs}{Microprocessors and Automatic Controls, Mechanical Measurements, Manufacturing Processes, Mechanical Workshop, Computer Programming}
\cvitem{Coding}{ C, C++, Python (Numpy, Pandas, Tensorflow, Requests, FEniCS), MATLAB, \LaTeX}
\cvitem{Hardware}{Arduino, ESP32, TIVA-C, Raspberry Pi 4B, Stepper and DC motor, I2C communication} %, Stepper Motor NEMA 17 with A3967 driver, DC motor (encoder read using Arduino Nano) with L298 driver, PCA9685 PWM driver, I2C}
\cvitem{Softwares}{SolidWorks, AutoCad, Adams, ROS, LABView, MS-Office, G-Suite, COMSOL, SOFA}
\cvitem{Languages}{English, Hindi, Marathi and learning French}

\vspace{-3mm}
\section{Position of Responsibility}
\cventry{Sept'21-Sept'22}{Water Polo Team Captain $|$ Aquatics, IIT Bombay Sports}{}{}{}{\textit{Synergized \textbf{13 member} institute \textbf{water polo} team in Inter IIT Aquatics meet 2022}\\
\vspace{-10pt}
\begin{itemize}
    \item Revamped the training process and the team structure after the 2 year COVID-19 pandemic break, as a result of which IIT Bombay reached \textbf{Semi Finals} after 6 years 
    \item Scouted for players, identified each player's strong points, scheduled regular team practices, matches and discussions for improving our game
\end{itemize}}

\cventry{Feb'20 -- Mar'21}{Events Convener $|$ Institute Sports Council $|$ IIT Bombay}{}{}{}{\textit{36-member team responsible for execution of sports events for 10K+ students and faculties}\\
\vspace{-10pt}
\begin{itemize}
    \item Ideated and organized first-ever \textbf{Virtual Run} with \boldmath{\(1.2K+\)} \textbf{runners} across the country to promote physical activity during the pandemic and raised INR 15K for \textbf{COVID relief} campaign by NGO-Goonj
    \item Conducted \textbf{Virtual Cup} for hostels through fantasy leagues and virtual General Championship; \textbf{India’s largest} 1-Day Online Chess tournament attracting \boldmath{\(550+\)} players (15 GMs) with prizes worth Rs70K
    \item Organized IIT Bombay’s annual sports fest \textbf{Aavhan} which attracted a lot of sports enthusiasts from \boldmath{\(150+\)} \textbf{colleges} across the country and \textbf{Blackcats Championship} a virtual fitness event for \boldmath{\(200+\)} inter IIT players across 14 sports
\end{itemize}}

\cventry{May'20 -- Jan'21}{Events Coordinator $|$ TechFest}{}{}{}{\textit{Asia's Largest Science and Technology Festival $|$ Events: \textbf{280+} $|$ Footfall: \textbf{175K+}}\\
\vspace{-10pt}
\begin{itemize}
    \item Assisted in HOPE - \boldmath{\(150+\)} virtual workshops on \textbf{Mental Health Awareness} in association with 10+ NGOs
    \item Collaborated with \boldmath{\(100+\)} \textbf{neurologists} and Mar de Somnis, a global non-profit organisation to train teachers in \boldmath{\(150+\)} \textbf{schools} to respond to epileptic seizures, under the aegis of HEAL, a social initiative by Techfest
\end{itemize}}

%----------------------------------------------------------------------------------------
%	INTERESTS SECTION
%----------------------------------------------------------------------------------------

\section{Extra Curriculars}
\subsection{Sports}
\cvitem{2022}{Secured \boldmath{\(2^{nd}\)} position in \(4 \times 100\) medley relay 55th Inter-IIT Aquatics meet and won \textbf{Gold} medal in medal in \(50\)m Butterfly and \textbf{Bronze} in \(4\times 50\)m Medley relay in Inter Hostel General Championship held at IIT Bombay}
\cvitem{2015--2016}{Represented School at \textbf{Zonal} and \textbf{State Level} Swimming competitions}
\cvitem{2016}{Participated in Inter-School \textbf{Football} League organized by YMCA}
\cvitem{2013}{Completed \textbf{State Level Sea Swimming} Competition (3km) held at Chivla beach, Malvan organized by Sindhudurg District Aquatic Association}

%\cvitem{2022}{Secured \(2^{nd}\) position in \(4 \times 100\) medley relay and 4th position in 400m Freestyle in 55th Inter-IIT Aquatics meet organised jointly by IIT Delhi and IIT Roorkee}
%
%\cvitem{2022}{Secured Gold medal in \(50\)m Butterfly and a Bronze medal in \(4\times 50\)m Medley relay in Inter Hostel General Championship held at IIT Bombay}
%
%\cvitem{2019}{Secured 4th position in 400m Freestyle in 54th Inter-IIT Aquatics meet organised jointly by IIT Kharagpur and IIT Bhubaneshwar}
%
%\cvitem{2015--16}{Selected for State Level Swimming competition for two consecutive years}
%
%\cvitem{2016}{Participated in Inter-School Football League organized by YMCA}
%
%\cvitem{Oct'15}{Represented school in Zonal Swimming competition organised by CBSE}
%
%\cvitem{Dec'13}{Completed State Level Sea Swimming Competition (3km) held at Chivla beach, Malvan organized by Sindhudurg District Aquatic Association}

\subsection{Others}
\cvitem{2022}{Selected \textbf{among 16 buddies} out of 115 applicants for Student Buddy Program which helps foreign exchange students breeze through their stay at our institute}
%\cvitem{2017}{Secured 95\% with Pravinya Shreni in Sanskrit Bhasha Parichay Pariksha organised by Sanskrit Bhasha Pracharini Sabha, Nagpur}
\cvitem{2014-2017}{Proficient in playing Tabla and have cleared exams with distinction organized by ‘Gandharva Akhil Bharatiya Mahavidyalay’}
\cvitem{2015-2017}{Member of Road Safety Patrol in Nagpur city for three consecutive years}
%\cvitem{2014}{Helped in spreading Cancer awareness as part of GCCI’s nationwide programme and helping with relief efforts for cancer sufferers}
%\cvitem{2012}{Secured 1st position in National level JetToy making competition in Maximum Distance category organised by SAEIndia}

%----------------------------------------------------------------------------------------

%\section{References}
%\cventry{}{\href{https://sites.google.com/site/vikramrentalahome/home}{Prof. Vikram Rentala}}{Research Supervisor}{}{}{\textit{Dept. of Physics, Indian Institute of Technology Bombay}\\Contact: \href{mailto:rentala@phy.iitb.ac.in}{rentala@phy.iitb.ac.in}}

%\cventry{}{\href{http://www.ncra.tifr.res.in/ncra/research/faculty-research-interests/faculty-profiles/preeti-kharb}{Dr. Preeti Kharb}}{Research Supervisor}{}{}{\textit{National Centre for Radio Astrophysics, Tata Institute of Fundamental Research (NCRA-TIFR)}\\Contact: \href{mailto:kharb@ncra.tifr.res.in}{kharb@ncra.tifr.res.in}}

%\cventry{}{\href{https://iitb.irins.org/profile/155686}{Prof. Kumar Rao}}{Course Professor: QM III, Elementary Particle Physics}{}{}{\textit{Dept. of Physics, Indian Institute of Technology Bombay}\\Contact: \href{mailto:kumar.rao@phy.iitb.ac.in}{kumar.rao@phy.iitb.ac.in}}

%\cventry{}{\href{http://home.iitb.ac.in/~yajnik/}{Prof. Urjit Yajnik}}{Course Professor: General Relativity, Adv. Gravity}{}{}{\textit{Dept. of Physics, Indian Institute of Technology Bombay}\\Contact: \href{mailto:yajnik@phy.iitb.ac.in}{yajnik@phy.iitb.ac.in}}

\end{document}